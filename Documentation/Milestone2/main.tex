\documentclass[12pt,a4paper]{article}

\usepackage[margin=1in]{geometry}

\usepackage{lmodern}
\usepackage[T1]{fontenc}
\usepackage[utf8]{inputenc}
\usepackage{microtype}
\usepackage{setspace}
\usepackage{parskip}

\usepackage{longtable}
\usepackage{booktabs}
\usepackage{array}

\usepackage{graphicx}
\usepackage{tikz}
\usetikzlibrary{positioning, arrows.meta, shapes.geometric, fit, backgrounds}

\usepackage{enumitem}

\usepackage{hyperref}
\hypersetup{colorlinks=true, linkcolor=blue, citecolor=blue, urlcolor=blue}

\usepackage[font=small,labelfont=bf]{caption}

\title{\vspace{-1.5cm}\textbf{Software Requirements Specification (SRS)}\\
\vspace{0.2cm}\large AI-powered Forex Trading Analysis Bot}
\author{Team: Hafsa Gull, Saad Nawaz, Shahbaz Ali \\
Course: CSC-225 — Software Engineering \\
Department of Computer Science, Namal University, Mianwali}
\date{Submission Date: \today}

\begin{document}
\maketitle
\thispagestyle{empty}
\newpage

\tableofcontents
\newpage

\section{Introduction}

\subsection{Purpose}
The purpose of this Software Requirements Specification (SRS) is to provide a
complete, unambiguous, and verifiable specification of the requirements
for the \textbf{AI-powered Forex Trading Analysis Bot}. This document explains the system's functionalities and constraints and serves as a reference for students, developers, and evaluators to understand the system requirements and to guide development, testing, and evaluation. 

\subsection{Scope}
The system is designed to support Forex traders by analyzing data across multiple timeframes and summarizing relevant fundamental (economic) events to produce AI-based directional suggestions with clear explanations. It is a decision-support prototype and does not execute trades automatically. The system focuses on integrating Forex price data and economic events, applying technical indicators, generating explainable analysis results, and presenting them through a web-based interface where users can customize indicators and export reports.

\subsection{Definitions, Acronyms, and Abbreviations}
\begin{description}[leftmargin=3cm,labelwidth=2.6cm]
    \item[Forex] Foreign Exchange market — global decentralized market for currency trading.
    \item[Currency Pair] A quotation of two currencies (e.g., EUR/USD).
    \item[Timeframe] Chart interval used for analysis (e.g., 4H, 1H, 15M).
    \item[Technical Indicator] Quantitative calculation based on price/time (e.g., RSI).
    \item[Fundamental Analysis] Assessment of economic events, news and macro data.
    \item[AI] Artificial Intelligence.
    \item[RP] Requirement Provider (domain expert for the project).
    \item[SRS] Software Requirements Specification.
    \item[UI] User Interface.
\end{description}
\subsection{References}
\begin{itemize}
    \item IEEE Std 830-1984, IEEE Guide to Software Requirements Specifications
    \item Forex Trading Analysis Bot Project Proposal
\end{itemize}
\subsection{Overview}
This SRS is organized per IEEE Std 830-1984 and contains:
\begin{itemize}
    \item \textbf{General Description} — product perspective, functions, environment.
    \item \textbf{Specific Requirements} — functional and non-functional requirements,
          external interfaces, assumptions and constraints, and traceability pointers.
    \item \textbf{Appendices} — use case and context diagrams plus other supporting informative items.
\end{itemize}

\section{General Description}

\subsection{Product Perspective}
The system is a standalone web-based application that assists traders in decision-making. It collects price data and economic events from external APIs, performs technical indicator analysis across multiple timeframes, and provides directional trading suggestions. Results are displayed through interactive graphical charts along with textual explanations of the bot's analysis. Users can customize indicators and thresholds, and the system recalculates and updates the charts and suggestions in real time, enabling personalized analysis and comparison. The system supports both guest and registered users, providing default analysis for guests and personalized features for registered users.

\section*{Context Diagram (Level-0 / Zero Diagram)}
\addcontentsline{toc}{section}{Context Diagram (Level-0 / Zero Diagram)}

\begin{figure}[h!]
\centering
\includegraphics[width=0.85\textwidth,keepaspectratio]{DFD.png}
\caption{Level-0 Context Diagram of the Forex Trading Analysis System}
\end{figure}



\subsection{Product Functions}
Major functions of the system include:
\begin{itemize}
\item \textbf{F1: User Management}

        - Support guestmode users with default analysis.

        - Support registered users with registeration, login and saved preferences, dashboards, and chat history.
    
    \item \textbf{F2: Data Collection} 
    
         - Collect historical and live Forex price data for selected currency pairs.

        - Retrieve economic events and fundamental market signals from external APIs.

    \item \textbf{F3: Indicator Calculation}

        - Compute technical indicators (e.g., RSI, MACD) across multiple timeframes.

        - Provide default calculations for guest users and customized calculations for registered users.

    \item \textbf{F4: AI Analysis}

        - Generate directional suggestions (UP, NEUTRAL, DOWN) with a confidence level for each timeframe.

        - Provide explanations showing how indicators influence AI suggestions.

    \item \textbf{F5: User Customization and Interaction}

        - Allow registered users to select indicators, adjust thresholds, and set preferred timeframes.

        - Recalculate results and update charts in real time based on user changes.

        - Enable users to query the bot for guidance or clarifications.

    \item \textbf{F6: Visualization and Reporting}

        - Display interactive charts with indicators, AI suggestions, and textual explanations.

        - Allow export of reports in PDF or CSV format.

        - Show default analysis for guests and personalized views for registered users.
\end{itemize}

\subsection{User Characteristics}

\begin{longtable}{|p{3cm}|p{4cm}|p{3cm}|p{4cm}|}
\hline
\textbf{User Type} & \textbf{Description} & \textbf{Experience / Skills} & \textbf{Constraints} \\
\hline
Beginner Trader & New to trading, occasional user & Limited understanding of technical indicators and Forex concepts & Needs simple interface, default analysis, textual bot explanations, and guidance \\
\hline
Expert Trader & Experienced in trading & Familiar with technical indicators and multiple timeframes & Requires customization options, saved preferences, detailed charts, minimal guidance \\
\hline
Registered User & Logged-in user with saved preferences & Varies (beginner to expert) & System must store preferences and chat history \\
\hline
Guest User & Access without an account & Varies & Default analysis only, no data saving \\
\hline
Developer / Maintenance & Responsible for system maintenance & Technical expertise in software and database management & Requires access to logs, configuration settings, and performance monitoring \\
\hline
\end{longtable}

\subsection{General Constraints}

The design and operation of the Forex Trading Analysis System are subject to several high-level constraints that influence implementation choices:

\begin{itemize}
    \item \textbf{External Data Dependency:} The system relies on third-party APIs to retrieve historical and live Forex prices and economic event data. API availability and rate limits may affect system performance and accuracy.
    
    \item \textbf{Data Storage:} All registered user data, including preferences, customized indicators, chat history, and exported reports, must be stored securely. Storage capacity and retrieval speed impose constraints on system scalability and responsiveness.
    
    \item \textbf{Real-Time Multi-User Calculations:} The system must support real-time recalculation of multiple technical indicators and AI-based analysis for multiple registered users simultaneously. 
    
    \item \textbf{Data Security:} Registered user information, authentication credentials, and personalized settings must be protected according to best practices for web application security.
\end{itemize}

\subsection{Assumptions and Dependencies}
\begin{itemize}
    \item At least one reliable Forex price API and one economic events API are available.
    \item It is assumed that data provided by third-party APIs is accurate, timely, and consistent.
    \item Users have basic familiarity with charting and trading terminology.
    \item Internet connectivity is present during normal operation.
    \item The backend server environment required to host the application and perform computations remains available and operational during system use.
\end{itemize}

\section{Specific Requirements}

Based on your requirements, here is the rewritten and consolidated list of Functional Requirements, with duplicates removed, economic calendar functions excluded (keeping only AI analysis integration), and organized in logical order:

\subsection{Functional Requirements (FRs)}

\subsubsection{FR-1: User Registration}

\paragraph{Introduction:} The system shall allow a new user to create an account to enable personalized features and data storage.

\paragraph{Inputs:}
\begin{itemize}
    \item Username
    \item Email address
    \item Password
    \item Confirm password
\end{itemize}

\paragraph{Processing:}
\begin{itemize}
    \item Validate that all required fields are filled.
    \item Verify the format of the email address.
    \item Ensure that the password and confirm password match.
    \item Check that the email address is not already registered.
    \item Encrypt the password before storage.
    \item Store validated user information in the database.
\end{itemize}

\paragraph{Outputs:}
\begin{itemize}
    \item Confirmation message on successful registration.
    \item Error message if registration fails.
\end{itemize}

\subsubsection{FR-2: User Login}

\paragraph{Introduction:} The system shall allow registered users to access the system using valid credentials.

\paragraph{Inputs:}
\begin{itemize}
    \item Registered email address
    \item Password
\end{itemize}

\paragraph{Processing:}
\begin{itemize}
    \item Verify that login fields are not empty.
    \item Authenticate user credentials against stored data.
    \item Create an active session upon successful authentication.
\end{itemize}

\paragraph{Outputs:}
\begin{itemize}
    \item Successful login message and system access.
    \item Error message for invalid credentials.
\end{itemize}

\subsubsection{FR-3: Login Failure Handling}

\paragraph{Introduction:} The system shall manage incorrect login attempts to ensure security.

\paragraph{Inputs:}
\begin{itemize}
    \item Incorrect email or password.
\end{itemize}

\paragraph{Processing:}
\begin{itemize}
    \item Detect and log invalid login attempts.
    \item Display an appropriate error message.
    \item Allow the user to retry login.
\end{itemize}

\paragraph{Outputs:}
\begin{itemize}
    \item Error message indicating login failure.
\end{itemize}

\subsubsection{FR-4: Guest Mode}

\paragraph{Introduction:} The system shall allow users to access the system without an account.

\paragraph{Inputs:}
\begin{itemize}
    \item User request to continue as guest.
\end{itemize}

\paragraph{Processing:}
\begin{itemize}
    \item Assign a temporary session to the guest user.
    \item Enable default indicators and settings.
    \item Disable personalized data saving features.
\end{itemize}

\paragraph{Outputs:}
\begin{itemize}
    \item Access to default analysis features.
    \item Notification of limited functionality.
\end{itemize}

\subsubsection{FR-5: User Logout}

\paragraph{Introduction:} The system shall allow users to securely exit the system.

\paragraph{Inputs:}
\begin{itemize}
    \item Logout request.
\end{itemize}

\paragraph{Processing:}
\begin{itemize}
    \item Terminate the active user session.
    \item Clear all session-related data.
\end{itemize}

\paragraph{Outputs:}
\begin{itemize}
    \item Logout confirmation.
    \item Redirection to the home or login page.
\end{itemize}

\subsubsection{FR-6: Access Control Based on User Type}

\paragraph{Introduction:} The system shall control feature access based on whether the user is a guest or a registered user.

\paragraph{Inputs:}
\begin{itemize}
    \item User session type (guest or registered).
\end{itemize}

\paragraph{Processing:}
\begin{itemize}
    \item Check user type at runtime.
    \item Restrict or enable features as per the user's type.
\end{itemize}

\paragraph{Outputs:}
\begin{itemize}
    \item Appropriate level of feature access.
\end{itemize}

\subsubsection{FR-7: Forex Historical Data Retrieval}

\paragraph{Introduction:} This function retrieves historical price data for user-selected currency pairs and timeframes from external APIs.

\paragraph{Inputs:}
\begin{itemize}
    \item Currency pair
    \item Timeframe
    \item Start date and end date
\end{itemize}

\paragraph{Processing:}
\begin{itemize}
    \item Validate that all inputs are provided.
    \item Send an API request to the primary Forex data provider.
    \item Parse the returned OHLC, volume, and timestamp data.
    \item Verify data covers the requested date range and timeframe.
    \item If data is incomplete, attempt to retrieve from a secondary API.
    \item Store parsed data in temporary memory or cache for calculation.
\end{itemize}

\paragraph{Outputs:}
\begin{itemize}
    \item Complete dataset matching the request parameters.
    \item Error message if data is unavailable from all sources.
\end{itemize}

\subsubsection{FR-8: Real-Time Data Fetching}

\paragraph{Introduction:} This function retrieves live market prices to update charts and AI calculations.

\paragraph{Inputs:}
\begin{itemize}
    \item Selected currency pair
    \item Selected timeframe
\end{itemize}

\paragraph{Processing:}
\begin{itemize}
    \item Verify the selected currency pair is supported.
    \item Send API request to retrieve the latest price/candle data.
    \item Validate timestamps and values for consistency.
    \item Handle API timeouts gracefully with configurable retries.
    \item Log API errors for diagnostics.
    \item Cache latest prices for immediate use by technical indicators.
\end{itemize}

\paragraph{Outputs:}
\begin{itemize}
    \item Latest market price data.
    \item Error message if data is unavailable.
\end{itemize}

\subsubsection{FR-9: Economic Events Data for AI Analysis}

\paragraph{Introduction:} The system shall retrieve historical economic events data relevant to selected currency pairs from external APIs for use as AI model features.

\paragraph{Inputs:}
\begin{itemize}
    \item Selected currency pair(s)
    \item Historical date range
    \item Event type filter (e.g., central bank decisions, inflation data)
\end{itemize}

\paragraph{Processing:}
\begin{itemize}
    \item Send API requests to the configured economic data provider.
    \item Parse returned data (event name, date/time, impact level, actual values).
    \item Filter events relevant to the selected currency pair(s).
    \item Store formatted event data for AI model integration.
    \item Handle API errors gracefully.
\end{itemize}

\paragraph{Outputs:}
\begin{itemize}
    \item Structured dataset of historical economic events.
    \item Error message if data cannot be retrieved.
\end{itemize}

\subsubsection{FR-10: API Response Validation}

\paragraph{Introduction:} Ensures data received from external APIs is complete, accurate, and usable.

\paragraph{Inputs:}
\begin{itemize}
    \item Raw data (JSON/CSV) from any external API.
\end{itemize}

\paragraph{Processing:}
\begin{itemize}
    \item Check for required fields (e.g., OHLC, timestamps).
    \item Detect and flag missing, null, or malformed data.
    \item Discard inconsistent entries.
\end{itemize}

\paragraph{Outputs:}
\begin{itemize}
    \item Validated, clean data structure.
    \item Error logs for invalid or missing data.
\end{itemize}

\subsubsection{FR-11: API Fallback Handling}

\paragraph{Introduction:} Provides continuity of data retrieval if the primary API fails.

\paragraph{Inputs:}
\begin{itemize}
    \item Primary API status (success/failure)
    \item Secondary API configuration
\end{itemize}

\paragraph{Processing:}
\begin{itemize}
    \item If primary API fails, automatically attempt the secondary API.
    \item Merge or replace failed data with the fallback source.
    \item Notify the user if fallback data is being used.
    \item Log all failover attempts.
\end{itemize}

\paragraph{Outputs:}
\begin{itemize}
    \item Complete dataset from an available API source.
    \item User notification in case of fallback activation.
    \item System log entries.
\end{itemize}

\subsubsection{FR-12: Technical Indicator Calculation}

\paragraph{Introduction:} The system shall compute selected technical indicators to support AI analysis and visualization.

\paragraph{Inputs:}
\begin{itemize}
    \item Validated price data (OHLC, volume)
    \item Selected currency pair and timeframe(s)
    \item User-selected indicators and parameters (or defaults for guests)
\end{itemize}

\paragraph{Processing:}
\begin{itemize}
    \item Retrieve the latest validated price data.
    \item Apply the corresponding calculation algorithm for each selected indicator.
    \item Process multiple timeframes simultaneously if requested.
    \item Handle missing data gracefully and flag unreliable results.
    \item Cache computed indicator values.
\end{itemize}

\paragraph{Outputs:}
\begin{itemize}
    \item Computed indicator values for the specified pair and timeframe(s).
    \item Warnings if calculations cannot be performed.
\end{itemize}

\subsubsection{FR-13: AI-Based Market Analysis}

\paragraph{Introduction:} The system shall generate directional suggestions (UP, NEUTRAL, DOWN) for currency pairs.

\paragraph{Inputs:}
\begin{itemize}
    \item Computed technical indicator values.
    \item Historical economic event data (FR-9).
    \item User-selected timeframe(s).
\end{itemize}

\paragraph{Processing:}
\begin{itemize}
    \item Feed input features (indicators + economic events) into the AI model.
    \item Calculate a directional suggestion and a confidence level.
    \item Generate a textual explanation linking key indicators and events to the suggestion.
\end{itemize}

\paragraph{Outputs:}
\begin{itemize}
    \item Directional suggestion for each timeframe.
    \item Associated confidence score.
    \item Explanation text.
\end{itemize}

\subsubsection{FR-14: Indicator Customization}

\paragraph{Introduction:} The system shall allow registered users to customize indicators, parameters, and thresholds.

\paragraph{Inputs:}
\begin{itemize}
    \item User-selected indicators.
    \item Custom parameter values (e.g., RSI period).
    \item Custom thresholds.
\end{itemize}

\paragraph{Processing:}
\begin{itemize}
    \item Store the user's custom indicator settings in the database.
    \item Apply customized calculations for charts and AI analysis.
    \item Validate parameter ranges.
    \item Update visualizations and calculations in real time.
\end{itemize}

\paragraph{Outputs:}
\begin{itemize}
    \item Confirmation of saved settings.
    \item Updated analysis results reflecting custom parameters.
\end{itemize}

\subsubsection{FR-15: Charting and Visualization}

\paragraph{Introduction:} The system shall display computed indicators and AI suggestions in interactive charts.

\paragraph{Inputs:}
\begin{itemize}
    \item Computed indicator values.
    \item AI directional suggestions and confidence.
    \item Selected currency pair and timeframe.
\end{itemize}

\paragraph{Processing:}
\begin{itemize}
    \item Plot indicators as overlays on price charts.
    \item Visually highlight AI suggestions and confidence levels.
    \item Allow zooming, scrolling, and timeframe switching.
    \item Render appropriate views for guest and registered users.
\end{itemize}

\paragraph{Outputs:}
\begin{itemize}
    \item Interactive charts with indicators and AI insights.
\end{itemize}

\subsubsection{FR-16: Report Generation and Export}

\paragraph{Introduction:} The system shall allow users to generate and export analysis reports.

\paragraph{Inputs:}
\begin{itemize}
    \item Selected currency pairs and timeframes.
    \item Associated indicators and AI suggestions.
    \item User-selected report format (PDF or CSV).
\end{itemize}

\paragraph{Processing:}
\begin{itemize}
    \item Collate analysis data into a structured report.
    \item Format the report according to the selected format.
    \item Validate data availability before export.
\end{itemize}

\paragraph{Outputs:}
\begin{itemize}
    \item Downloadable report file.
    \item Error message if export fails.
\end{itemize}

\subsubsection{FR-17: User Preferences Management}

\paragraph{Introduction:} The system shall allow registered users to save, load, and manage settings.

\paragraph{Inputs:}
\begin{itemize}
    \item User-selected indicators, thresholds, and UI preferences.
    \item Request to save, load, or reset preferences.
\end{itemize}

\paragraph{Processing:}
\begin{itemize}
    \item Store user preferences securely in the database.
    \item Apply saved preferences upon user login or request.
    \item Allow resetting to default settings.
\end{itemize}

\paragraph{Outputs:}
\begin{itemize}
    \item Confirmation of saved preferences.
    \item Updated UI and analysis configuration.
\end{itemize}

\subsubsection{FR-18: Data Caching}

\paragraph{Introduction:} The system shall cache API responses and computed values to improve performance.

\paragraph{Inputs:}
\begin{itemize}
    \item API response data.
    \item Computed indicator results.
\end{itemize}

\paragraph{Processing:}
\begin{itemize}
    \item Store frequently used data in cache with a Time-to-Live (TTL).
    \item Invalidate cache when new data arrives.
    \item Retrieve cached data for repeated queries before calling APIs.
\end{itemize}

\paragraph{Outputs:}
\begin{itemize}
    \item Faster data retrieval.
    \item Reduced number of external API calls.
\end{itemize}

\subsubsection{FR-19: Historical Analysis Storage}

\paragraph{Introduction:} The system shall maintain a history of AI suggestions and analysis for review.

\paragraph{Inputs:}
\begin{itemize}
    \item Computed indicators and AI suggestions.
    \item Timestamp of analysis.
\end{itemize}

\paragraph{Processing:}
\begin{itemize}
    \item Store historical analysis results with timestamps.
    \item Allow registered users to query their past analyses.
\end{itemize}

\paragraph{Outputs:}
\begin{itemize}
    \item Accessible logs of historical analysis for registered users.
\end{itemize}

\subsubsection{FR-20: Multi-Timeframe Analysis Aggregation}

\paragraph{Introduction:} The system shall aggregate AI results across timeframes to provide a consolidated market view.

\paragraph{Inputs:}
\begin{itemize}
    \item AI suggestions from multiple timeframes.
\end{itemize}

\paragraph{Processing:}
\begin{itemize}
    \item Compare and combine results from different timeframes.
    \item Resolve conflicts using weighted logic.
    \item Generate an overall market trend or bias.
\end{itemize}

\paragraph{Outputs:}
\begin{itemize}
    \item Consolidated market direction and confidence.
    \item Explanation showing contributing timeframes.
\end{itemize}

\subsubsection{FR-21: Backtesting}

\paragraph{Introduction:} The system shall allow testing of indicator strategies against historical data.

\paragraph{Inputs:}
\begin{itemize}
    \item Selected historical date range.
    \item User-defined indicator strategy.
\end{itemize}

\paragraph{Processing:}
\begin{itemize}
    \item Apply the strategy calculations on historical price data.
    \item Compare strategy signals with actual market outcomes.
    \item Generate performance metrics (e.g., win rate).
\end{itemize}

\paragraph{Outputs:}
\begin{itemize}
    \item Backtesting report with performance metrics.
    \item Graphical comparison of strategy performance.
\end{itemize}

\subsubsection{FR-22: Alert System}

\paragraph{Introduction:} The system shall notify users of specific market conditions.

\paragraph{Inputs:}
\begin{itemize}
    \item Real-time price and indicator data.
    \item User-defined alert conditions (price levels, indicator thresholds).
\end{itemize}

\paragraph{Processing:}
\begin{itemize}
    \item Monitor market data against user conditions in real-time.
    \item Trigger notifications when conditions are met.
    \item Support user preferences for alert types.
\end{itemize}

\paragraph{Outputs:}
\begin{itemize}
    \item Real-time alerts via the system UI.
    \item Log of triggered alerts.
\end{itemize}

\subsubsection{FR-23: Interactive AI Chat}

\paragraph{Introduction:} The system shall provide a chat interface for users to query AI suggestions and request explanations.

\paragraph{Inputs:}
\begin{itemize}
    \item User queries in natural language.
    \item Current analysis context.
\end{itemize}

\paragraph{Processing:}
\begin{itemize}
    \item Process user queries to identify intent.
    \item Retrieve relevant market analysis data.
    \item Generate human-readable, context-aware explanations.
    \item Maintain chat history for registered users.
\end{itemize}

\paragraph{Outputs:}
\begin{itemize}
    \item Interactive, explanatory responses.
    \item Persistent chat history for registered users.
\end{itemize}

\subsubsection{FR-24: Admin Dashboard}

\paragraph{Introduction:} The system shall provide administrative tools for monitoring and management.

\paragraph{Inputs:}
\begin{itemize}
    \item Admin login credentials.
    \item System logs and performance data.
\end{itemize}

\paragraph{Processing:}
\begin{itemize}
    \item Display system health and performance metrics.
    \item Track API usage and error rates.
    \item Provide basic user account management.
\end{itemize}

\paragraph{Outputs:}
\begin{itemize}
    \item Admin dashboard with monitoring visuals.
    \item System audit and troubleshooting logs.
\end{itemize}

\subsubsection{FR-25: Security and Session Management}

\paragraph{Introduction:} The system shall manage session security and data protection.

\paragraph{Inputs:}
\begin{itemize}
    \item User login/session data.
\end{itemize}

\paragraph{Processing:}
\begin{itemize}
    \item Enforce secure session timeouts.
    \item Encrypt sensitive data and credentials.
    \item Implement secure authentication practices.
\end{itemize}

\paragraph{Outputs:}
\begin{itemize}
    \item Secure user sessions.
    \item Protected user data.
\end{itemize}

\subsubsection{FR-26: Data Privacy Management}

\paragraph{Introduction:} The system shall provide tools for managing user data in compliance with privacy regulations.

\paragraph{Inputs:}
\begin{itemize}
    \item User data export or deletion requests.
\end{itemize}

\paragraph{Processing:}
\begin{itemize}
    \item Provide data export functionality (JSON/CSV).
    \item Implement complete data deletion upon user request.
\end{itemize}

\paragraph{Outputs:}
\begin{itemize}
    \item Exported data packages.
    \item Confirmation of data deletion.
\end{itemize}

\subsubsection{FR-27: Multi-Currency Pair Comparison}

\paragraph{Introduction:} The system shall allow side-by-side comparison of multiple currency pairs.

\paragraph{Inputs:}
\begin{itemize}
    \item Selected currency pairs.
    \item Comparison parameters and timeframe.
\end{itemize}

\paragraph{Processing:}
\begin{itemize}
    \item Synchronize charts for the selected pairs.
    \item Compute and display correlation metrics.
\end{itemize}

\paragraph{Outputs:}
\begin{itemize}
    \item Multi-pair comparison interface.
    \item Correlation analysis summary.
\end{itemize}

\subsubsection{FR-28: API Management and Monitoring}

\paragraph{Introduction:} The system shall monitor API usage and manage keys for performance and cost control.

\paragraph{Inputs:}
\begin{itemize}
    \item API keys and rate limits.
    \item API usage statistics.
\end{itemize}

\paragraph{Processing:}
\begin{itemize}
    \item Track API usage per user and system-wide.
    \item Enforce quotas and manage key rotation.
    \item Provide analytics on API performance and cost.
\end{itemize}

\paragraph{Outputs:}
\begin{itemize}
    \item API usage and cost dashboard.
    \item Quota management interface.
\end{itemize}

\subsubsection{FR-29: Tutorial and Help System}

\paragraph{Introduction:} The system shall provide context-sensitive help and tutorials.

\paragraph{Inputs:}
\begin{itemize}
    \item User request for help.
    \item UI context (current page/feature).
\end{itemize}

\paragraph{Processing:}
\begin{itemize}
    \item Display interactive tutorials for new users.
    \item Show tooltips and detailed explanations for complex features/terms.
\end{itemize}

\paragraph{Outputs:}
\begin{itemize}
    \item On-demand help content.
    \item Step-by-step guides.
\end{itemize}

\subsubsection{FR-30: Error Handling and User Notifications}

\paragraph{Introduction:} The system shall notify users of errors or important system events.

\paragraph{Inputs:}
\begin{itemize}
    \item System errors (API failures, calculation errors).
    \item Invalid user inputs.
\end{itemize}

\paragraph{Processing:}
\begin{itemize}
    \item Detect and log errors.
    \item Display user-friendly notifications describing the issue.
    \item Suggest corrective actions where possible.
\end{itemize}

\paragraph{Outputs:}
\begin{itemize}
    \item Informative error or warning messages.
    \item System logs for diagnostics.
\end{itemize}


\subsection{Non-Functional Requirements (NFRs)}
Non-functional requirements provide measurable criteria.

\begin{longtable}{p{3cm} p{10.8cm}}
\toprule
\textbf{ID} & \textbf{Requirement} \\
\midrule
NFR-1 (Performance) & The system shall serve a standard analysis request (precomputed indicators + AI inference) and render the results on the dashboard within \textbf{2 seconds} under expected prototype load. \\
\addlinespace
NFR-2 (Availability) & The system shall be available to users during typical working hours (weekdays) with reasonable uptime; prototype uptime target is \textbf{> 95\%} during demo periods. \\
\addlinespace
NFR-3 (Reliability) & The system shall gracefully degrade when an external data API fails, showing cached results and a clear message indicating stale data. \\
\addlinespace
NFR-4 (Security) & The system shall never store raw user credentials for external services; any credentials shall be stored only using secure secrets management if needed. \\
\addlinespace
NFR-5 (Usability) & The user interface shall be intuitive for traders with basic knowledge; initial user testing should yield an average SUS-like subjective usability score above a defined baseline (to be collected during evaluation). \\
\addlinespace
NFR-6 (Explainability) & For every AI suggestion, the system shall present at least the top three contributing features and an automatically generated natural-language explanation. \\
\addlinespace
NFR-7 (Traceability) & All functional requirements shall have unique IDs and be traceable to use cases and test cases. \\
\bottomrule
\end{longtable}

\subsection{External Interface Requirements}
\subsubsection{User Interface}
The system shall provide a web-based dashboard for:
\begin{itemize}
    \item Selecting currency pairs and timeframes.
    \item Viewing chart(s) with indicator overlays.
    \item Displaying AI signals and explanations.
    \item Managing user-defined indicators and exporting reports.
\end{itemize}

\subsubsection{Software Interfaces}
\begin{itemize}
    \item The system shall call a Forex price API to fetch historical and live price data (REST/HTTPS).
    \item The system shall call an economic events API (REST/HTTPS) to retrieve scheduled events.
    \item The system shall expose an internal API (for integration/testing) to request analysis programmatically.
\end{itemize}

\subsubsection{Hardware Interfaces}
No specialized hardware is required for prototype operation beyond a standard server/VM for the backend and user device for the frontend.

\subsection{Traceability}
Each requirement ID in Sections above will be referenced in test plans and use case mappings in the project repository. Example mapping:
\begin{itemize}
    \item \textbf{UseCase: Analyze Market} $\rightarrow$ FR-1, FR-4, FR-5, FR-6, FR-7, FR-9.
    \item \textbf{UseCase: Apply Custom Indicator} $\rightarrow$ FR-8, FR-9, FR-10.
\end{itemize}

\section*{System Architecture Diagram}
\addcontentsline{toc}{section}{System Architecture Diagram}

\begin{figure}[h!]
\centering
\includegraphics[height=0.7\textheight,keepaspectratio]{UseCase.png}
\caption{Overall System Architecture of AI-powered Forex Trading Analysis Bot}
\end{figure}



\section{Appendices}

\subsection{Appendix A: Use Case Descriptions (brief)}
\begin{description}[leftmargin=!, labelwidth=\widthof{\bfseries Use Case: Analyze Market}]
    \item[\textbf{Use Case: Analyze Market}] Primary actor: Trader. Precondition: Currency pair and timeframe selected. Postcondition: AI directional suggestion and explanation displayed; user may export a report.
    \item[\textbf{Use Case: Generate AI Signal}] Primary actor: Trader / system (automated). Precondition: Valid feature vector available. Postcondition: Directional signal + confidence and explanation produced.
    \item[\textbf{Use Case: Apply Custom Indicators}] Primary actor: Trader. Precondition: UI supports custom indicator definitions. Postcondition: Custom indicator computed; results compared against AI signals.
\end{description}


 
 



\end{document}